%%%Glossary File:

\newglossaryentry{AFM}{name=AFM,description={Atomic force
microscopy. High resolution type of scanning probe microscopy. It has a
lateral resolution of < 1\nm{} and height resolution of < 1\angm{}.
\citet{garcia_dynamic_2002} contains a very interesting review using AFM in
polysaccharides. A more general study about dynamical modes can be found in
\citet{rief_single_1997} }}
 

\newglossaryentry{Lp}{name=$\ell_p$,description={Persistence length. Distance
over which the angular correlation decreases to $1/e$ of its initial value.
Equivalently, typical length scale for the decay of tangent-tangent
correlations: $\langle \textbf t(s)\cdot \textbf t(s') \rangle=
\exp[-(s-s')/\ell_p]$ , where s is the arc-length of the chain, and $\textbf
t(s)=\partial{\textbf r(s)}/\partial s$ is the tangent vector.\\ 
$\lp{}=2\kbend{}/((d-1)\kbolt{}T)$ in d-dimensional space where $\kbend{}$ is
the bending stiffness of the chain \citep{frey_viscoelasticity_1998}
}}         

\newglossaryentry{Lc}{name=$L_c$,description={Contour length. The maximum
end-to-end distance of a linear polymer chain. For a single-strand polymer
molecule,  this usually means the end-to-end distance of the chain extended  to
the all-trans conformation. For chains with complex structure, only an 
approximate value of the contour length may be accessible.  }}
\newglossaryentry{mesh}{name=$\xi_m$,description={Mesh-size}} 
 
\newglossaryentry{Le}{name=$L_e$,description={Deflection or entanglement
length. In the tube model picture: typical distance between two collision points
of a ``test-polymer'' with its surrounding
chains\citep{frey_viscoelasticity_1998}. If one approximates the effect of the surrounding medium by a cylindrical tube of diameter d (of the
order of magnitude of the mesh size) the entanglement length is given by
Odijk’s estimate \citep{odijk_statistics_1983}      
$L_e\simeq d^2 \lp{}$     
}}

\newglossaryentry{TEM}{name=TEM,description={Tomography electron microscopy
}}

\newglossaryentry{SEM}{name=SEM,description={Scanning electron microscopy
}}

\newglossaryentry{confocal}{name=confocal microscopy,description={Confocal
microscopy }}

\newglossaryentry{graph}{name=graph,description={graph}}