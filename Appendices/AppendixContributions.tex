\chapter{Publications and Open Source Contributions}

\label{Appendix-Contributions}

\lhead{Contributions} % This is for the header on each page -

\section{Publications}

\begin{enumerate}[label=\bfseries\alph*)]
\item Isotropic Wavelets.
\item TEM-SAXS comparison.
\item Network skeletonization.
\end{enumerate}

\section{Open Source Contributions}%
\label{sec:contributions}

\subsection{ITKIsotropicWavelets}%
\label{sub:contribution_itk}
For Insight ToolKit: ITKIsotropicWavelets External Module \url{https://github.com/phcerdan/ITKIsotropicWavelets/}

  This contribution to ITK provides a multi-resolution framework based on a steerable wavelet pyramid \cite{simoncelli_steerable_1995} with all the isotropic mother wavelets available in the literature \cite{held_steerable_2010, pad_vow:_2014}. It also couple the pyramid with the Riesz function \cite{chenouard_3d_2012} for directional analysis. Specially useful for the study of textures \cite{depeursinge_steerable_2017} and phase analysis \cite{held_steerable_2010}.

In this thesis, this module work has been used to aid the segmentation of the networks as a pre-processing step before binarization. It improves the signal to noise ratio, helping to choose an appropriate binary threshold to capture the network.

I have contributed with more than 20.000 lines of c++ code and documentation. This work has been
integrated into ITK as an external module since ITK version 4.13, this allow the user to build the module directly from the ITK.
Also wrappings for the c++ code to use it from python are available.

\subsection{DGtal: Critical Kernels Framework}%
\label{sub:contribution_dgtal}

Pull request with review of the contribution:
\url{https://github.com/DGtal-team/DGtal/pull/1147}

The contribution to DGTal implements the skeletonization (or thinning) recently proposed in the
literature \cite{couprie_3d_2015, bertrand_parallel_2017} based on critical kernels frameworks, to perform thinning conserving the topology of the original object.

Ready to use tools using this framework can be found in:
\url{https://github.com/DGtal-team/DGtalTools}

The work was reviewed, accepted and merged in the library since version 0.9.4.

I have contributed with more than 7000 lines of c++ code, documentation and scripts for applications.



