\chapter{Publications and Open Source Contributions}

\label{Appendix-Contributions}

\lhead{Contributions} % This is for the header on each page -

\section{Open Source Contributions}%
\label{sec:contributions}

\subsection{ITKIsotropicWavelets}%
\label{sub:contribution_itk}
For Insight ToolKit: ITKIsotropicWavelets External Module \url{https://github.com/phcerdan/ITKIsotropicWavelets/}

  This contribution to ITK provides a multi-resolution framework based on a steerable wavelet pyramid \cite{simoncelli_steerable_1995} with all the isotropic mother wavelets available in the literature \cite{held_steerable_2010, pad_vow:_2014}. It also couple the pyramid with the Riesz function \cite{chenouard_3d_2012} for directional analysis. Specially useful for the study of textures \cite{depeursinge_steerable_2017} and phase analysis \cite{held_steerable_2010}.

In this thesis, this module work has been used to aid the segmentation of the networks as a pre-processing step before binarization. It improves the signal to noise ratio, helping to binarize the image to capture the network.

I have contributed with more than 20.000 lines of c++ code and documentation. This work has been
integrated into ITK as an external module since ITK version 4.13, this allow the user to build the module directly from the ITK.
Also wrappings for the c++ code to use it from python are available.

\subsection{Radial Intensity FFT}%
\label{sub:radial_intensity_fft}
The code used in \autoref{Chapter-Wavelets}, including a graphical interface for the study of the radial intensity of a Fourier Transform
is available in \url{https://github.com/phcerdan/FFTRadialIntenstiy}.

The author has contributed to around 3000 lines of c++ code and 1000 python code, including a notebook that can be used to reproduce exactly the work-flow and figures shown in \autoref{Chapter-TEMSAXS}, find the notebook here:

\noindent\url{https://github.com/phcerdan/FFTRadialIntenstiy/tree/master/resultsPaper/figures_jupyternb}

\subsection{FFT Radial Intensity}%
\label{sub:contribution_fft_radial_intensity}

Software used in \autoref{Chapter-TEMSAXS} to create a radial intensity plot from a Fourier Transformed image.
It gives the average intensity of all the pixels of an FFT image at a certain frequency. If the FFT image is shifted, i.e. with the zero frequency pixel at the center of the image,
it is equivalent to averaging the value of the pixels that intersect the perimeter of concentric rings of different radius, with the radius representing the modulo of the frequency.
Useful to compare images with scattering data. It can be used as console script or with a graphical interface.

\url{https://github.com/phcerdan/FFTRadialIntenstiy}

\subsection{DGtal: Critical Kernels Framework}%
\label{sub:contribution_dgtal}

Pull request with review of the contribution:
\url{https://github.com/DGtal-team/DGtal/pull/1147}

The contribution to DGTal implements the skeletonization (or thinning) recently proposed in the
literature \cite{couprie_3d_2015, bertrand_parallel_2017} based on critical kernels frameworks, to perform thinning conserving the topology of the original object.

Ready to use tools using this framework can be found in:
\url{https://github.com/DGtal-team/DGtalTools}

The work was reviewed, accepted and merged in the library since version 0.9.4.

I have contributed with more than 7000 lines of c++ code, documentation and scripts for applications.

\subsection{Spatial Graph Extraction}%
\label{sub:spatial_graph_extraction}

The code for the spatial graph extraction is available with a Mozilla Public License (open source) in
\url{https://github.com/phcerdan/object_to_spatial_graph}.

This includes scrips in c++ and python to perform the image pipeline, the skeletonization script using DGtal (see \autoref{sub:contribution_dgtal}), converting the raw graph to a spatial graph, computing graph properties and statistical distributions, as well as the code for generating the figures in \autoref{sec:sg_results}. It uses ITK, VTK, Boost, DGtal and proxTV libraries.

Python scripts for plotting and fitting statistical distributions (see \autoref{fig:actin_thin}) can be found in the mentioned repository. Concretely here:

\noindent\url{https://github.com/phcerdan/object_to_spatial_graph/tree/master/src/python-scripts}

I have contributed with around 8000 lines of c++ and python code.

\subsection{Open Science}%
\label{sub:open_science}

All the work done in this thesis is openly available, and can be reproduced and modified by anyone.
I have tried to split my work into chunks and contribute them into existing libraries to maximize its re-usability.

I also provide notebooks with the analysis, data and plots created in \autoref{Chapter-TEMSAXS}.

The notebook of \autoref{Chapter-Image} is not provided due to the analysis was done between two computers, with the thin/skeletonization process was carried in a workstation with more memory.
But the chapter should provide enough information about the pipeline and the details to reproduce it.

Users can contact me in any of these public github repositories with any inquiry or issue, and can contribute adding new features.







