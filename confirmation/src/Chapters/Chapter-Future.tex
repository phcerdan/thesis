% Chapter Template

\chapter{Schedule and future work} % Main
% chapter title

\label{Chapter-future} % Change X to a consecutive number; for
% referencing this chapter elsewhere, use \ref{ChapterX}

\lhead{Chapter 3. \emph{Schedule and future work}} % Change X to a consecutive
% number; this is for the header on each page - perhaps a shortened title

%----------------------------------------------------------------------------------------
%	SECTION 1
%----------------------------------------------------------------------------------------

\section{Schedule}

\begin{description}
  \item[0-6 months] Landing. Literature overview. Planification and research on
  computational tools.
  \item[6-12 months] Reconstruct-network software development.
   Access to image analysis tools. Avizo and Matlab.
  \item[12-18 months] Selection/development of proper model in order to simulate
  the strain of the network. Think about how simulate long time behavior in the
  networks. Further exploration of image analysis.
  \item[18-24 months] Explore models and other theoretical frameworks.
  Scaling and renormalization group theory, graph theory to explore strain
  pathways and other networks properties. Plausible experimental experience if
  required.
  \item[24-30 months] Put all together with a solid bottom-up and up-bottom
   approach convergent in the mesoscale. Image analysis of real networks,
   dynamics simulation and in-silico reconstructed networks.
   \item[30-36 months] Conclusions and writing of the
   thesis report.
   
\end{description}
Stimated date of submission: January $2016$
\subsection{Funding:}
This PhD project is funded by the MacDiarmid and Riddet
institute. 

%-----------------------------------
%	SUBSECTION 1
%-----------------------------------
\section{Future directions and interesting questions}

The field is full of opportunities and in continuous development. The exact
direction to follow will rely on further discussion with my supervisor, and the necessities
of the project from which I am receiving funding. The structure of that project,
and my role on it is clear.

\subsection{``Mesocule'' project}
My PhD fundings come from a project which aim is to explore the hierarchical
structures exhibited by many biomaterials. These structured are organized on
multiple length-scales, with emergent properties not shown by individual
components. The mesoscale, the scale between the bottom and the top, may be  a
important source of new knowledge of these biomaterials.

The project is formed by
$3$ PIs from Victoria, Canterbury and Massey universities, $2$ Post-docs
 and $2$ PhD. students. We have meetings every $6$ months to report advances and
 to choose new directions.
I will briefly summarize the proposed structure of the project: using optical
tweezers, one of the members is able to  measure force
extension curves of a single biopolymer chain. This expertise can also be  used
to measure protein fibrils that other member of the group can create,  different
properties of the fibril can be tweaked in the growth process. These protein
fibrils can be assembled into gelled materials.

Then, having measured the mechanical properties of these engineered
fibrils, and with the network architecture information, gathered after doing
image analysis on microscopy images of the network, we will follow a
bottom-up approach \citep{brown_multiscale_2009,schuster_investigating_2012} to
model and compute the mechanical properties of the whole networks from the single-fibril properties.
Then we will compare those simulated results to
bulk rheology measurements in the macro-scale, done by a fourth member.
Also the developed software about reconstructing network will permit us to 
study how the architecture of the network affect the bulk properties of the material.
 





%-----------------------------------
%	SUBSECTION 2
%-----------------------------------

\subsection{Long time behaviour: network quakes and aging}
There are reports in the literature \citep{kajiya_slow_2013}, and also by group
mates\citep{vincent_micro-rheological_2013}, that at long times, physical
biopolymer networks can be affected by sudden de-correlations that occur over a
few tenths of a second. The more plausible conjecture is that such quakes are associated with the release of
chain constraint due to unbinding of cross-links.  This is reported
experimentally in our group  with micro-rheology DWS techniques.
Current models of networks, such as the Glassy WLC model
\citep{kroy_glassy_2007}, cannot predict this behaviour \citep{vincent_micro-rheological_2013}.  A
computer simulation may shed some light on the topic.

It is also interesting the aging phenomena: the slow change of the physical
properties over time. This behaviour is related with non-ergodic systems and
with glassy systems, where long orders correlations(solids) no longer exists,
but there is still some local order.\citep{cipelletti_slow_????}

\subsection{Non-affine networks}
As discussed in section \ref{intro-strainstiffening}, at low densities of
cross-links the network behaves in a non-affine way, where the chain
straining depends on the local heterogeneity of the media.
This non-affinity is governed by the micro-architecture of the network
\citep{head_deformation_2003,wilhelm_elasticity_2003,huisman_three-dimensional_2007,chandran_affine_2006}.
So, in collaboration with a member of my group, an approach to study the
dynamics modes of non-affine biopolymer networks for different strain
frequencies might be pursued.

\subsection{Cluster of nodes as mesoscale objects}
The resolution of image analysis is not enough to provide the network
architecture in some high density areas of the material. These clusters could
then be treated as new mesocule objects, similar to nodes, but with different
properties to those that represent cross-links. This scaling approach could be
studied with the renormalization group theory, which has been successful to
explain scaling behavior of polymers in the past.\citep{gennes_scaling_1979}


\subsection{Computational methods}  
The last Nobel prize in Chemistry was given "for the development of multi-scale
models for complex chemical systems" \citep{nobel:chemistry2013} These methods
allow the simulation of protein structures,  docking of drugs in cells and
multiple complex chemical reactions.

These coarse-graining methods study really complex
systems \citep{de_pablo_coarse-grained_2011}, impossible to compute with  the
fine detail potentials of molecular dynamics, but with effective potentials 
derived from scaling, coarse graining and renormalization group theories. These
effective potentials contain the information of the lower scales and can be
tweaked if that information changes, but it is not necessary to simulate all the
degrees of freedom of the fine structure. The communication over scales is the
responsible of the success of the method, where changes in the lower scales are
highly correlated with scales larger by several orders of magnitude. Advanced
Monte Carlo methods are in continuous development to achieve this
\citep{karayiannis_novel_2002}.

There are plenty of other methods to explore, finite element methods (FEM),
conjugate gradients, and a plethora of lattice models seem the most popular in
the literature.  A future review about this topic will be needed.


% ---------------------------------------------------------------------------------------

% \section{Future work}

