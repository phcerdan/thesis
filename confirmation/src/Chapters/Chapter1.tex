% Chapter Template

\chapter{Introduction} % Main chapter title

\label{Introduction} % Change X to a consecutive number; for referencing
% this chapter elsewhere, use \ref{ChapterX}
%Default was: Chapter 1 .\emph{Introduction} 
\lhead{\emph{Introduction}} % Change X to a consecutive number;
% this is for the header on each page - perhaps a shortened title

%----------------------------------------------------------------------------------------
%	SECTION 1
%----------------------------------------------------------------------------------------


    
    

\section{Soft matter and biopolymers}
Soft matter, also known as complex fluids, is a subfield of condensed
matter, comprising systems which:
\begin{enumerate*}[label=\bfseries\alph*)]
 
\item organize at many different length scales, into
many different forms and it classifies as something in between the
ordered solids and disordered liquids.\
\item its conformation is heavily
influenced by thermal fluctuations in the energy scale comparable to the room
temperature. This characteristic allows conformational changes where complex
behaviour can occur, life as the outermost example.
\end{enumerate*}

Soft matter includes liquids, colloids, polymers, foams, gels, granular
materials, surfactants, liquids crystals and some biological materials.
Pierre-Gilles de Gennes, who has been called the ``founding father of soft
mater'' received the Nobel Prize in physics in $1991$ ``for discovering that 
methods developed for studying order phenomena in simple systems can be 
generalized to more complex forms of matter, in particular to liquid  crystals
and polymers''\citep{de_gennes_pierre-gilles_????}

The biological soft materials spam as well different length scales. From sugar
chains structures to active gels inside cells. In the
lowest scale, we find the biopolymers, which are classified as polysaccharides
(cellulose, pectin),  polynucleotides (RNA, DNA) or polypeptides (proteins).

Most of biopolymers are considered semi-flexible. Something in between
rigid rods and completely flexible (loose) chains. Suppose both ends of a chain
are fixed in space. If the chain is rigid -a rod-, there is only one possible conformation
(if we assume there is rotational symmetry), if the chain is flexible, there
are a lot possible conformations or paths between the two ends. In the later
case, the system will be dominated by entropic effects.

A polymer chain can be characterized as an ideal chain, where there aren't
interaction between monomers separated by many bonds. In ideal chains the free
energy of the chain is completely entropic. The worm-like chain is an ideal
model valid for stiff polymers, are shows an excellent agreement with stiff
biopolymers as double stranded DNA.
 pg 88 Rubinstein:
R
P
f= entropic spring.

In a real chain, where correlation between monomers are included.
For example with the excluded volume term because other monomers is included.


Talk about Glassy worm like chain in networks


\gls{Lp}  
 (\citet{storm_nonlinear_2005})
\citet{stein_algorithm_2008}



\subsection{Subsection 1}





%----------------------------------------------------------------------------------------
%	SECTION 2
%----------------------------------------------------------------------------------------

\section{Complex Networks}

\subsection{Subsection 2}