% Chapter Template

\chapter{Introduction} % Main chapter title

\label{Introduction} % Change X to a consecutive number; for referencing
% this chapter elsewhere, use \ref{ChapterX}
%Default was: Chapter 1 .\emph{Introduction} 
\lhead{\emph{Introduction}} % Change X to a consecutive number;
% this is for the header on each page - perhaps a shortened title

%----------------------------------------------------------------------------------------
%	SECTION 1
%----------------------------------------------------------------------------------------


    
    

\section{Soft matter and biopolymers}
Soft matter, also known as complex fluids, is a subfield of condensed
matter, comprising systems which:
\begin{enumerate*}[label=\bfseries\alph*)]
 
\item organize at many different length scales, into
many different forms and it classifies as something in between the
ordered solids and disordered liquids.\
\item its conformation is heavily
influenced by thermal fluctuations in the energy scale comparable to the room
temperature. This characteristic allows conformational changes where complex
behaviour can occur, life as the outermost example.
\end{enumerate*}

Soft matter includes liquids, colloids, polymers, foams, gels, granular
materials, surfactants, liquids crystals and some biological materials.
Pierre-Gilles de Gennes, who has been called the ``founding father of soft
mater'' received the Nobel Prize in physics in $1991$ ``for discovering that 
methods developed for studying order phenomena in simple systems can be 
generalized to more complex forms of matter, in particular to liquid  crystals
and polymers''\citep{de_gennes_pierre-gilles_????}
 
The biological soft materials spam as well different length scales. From sugar
chains structures to active gels inside cells. In the
lowest scale, we find the biopolymers, which are classified as polysaccharides
(cellulose, pectin),  polynucleotides (RNA, DNA) or polypeptides (proteins).
 
Most of biopolymers are considered semi-flexible. Something in between
rigid rods and completely flexible (loose) chains.  To formalize this
distinction we need to introduce $2$ parameters that characterize a polymer
chain: the \emph{persitence length} \gls{Lp} which is the typical length scale
for the decay of tangent-tanget correlations, and the \emph{countour length}
\gls{Lc}, which is defined as the maximum end-to-end distance of a linear
polymer chain.

A chain is considered flexible when $\ell_p<<L_c$, and rigid when the opposite
holds. Completely flexible chains exhibit a purely entropic elastic
response, and rigid filaments display no entropic, but purely enthalpic
response. Semi-flexible biopolymers have a similar magnitude of $\ell_p$ and
$L_c$. These kind of filaments does not form loops or knots, but they are
flexible enough to have thermal bending
fluctuations\citep{storm_nonlinear_2005}. They behave like rods scales smaller
than $\ell_p$ and like random coils in larger scales where the entropic behavior
dominates.

Single semiflexible chains can be modeled with great success with the
\emph{worm-like chain} (WLC) also known as \emph{Kratky-Porod} model.
\citep{rubinstein_polymer_2003, schuster_hierarchical_2011}. This model has
became a standard in the field thanks to the agreement with experiments in low
and large scales.
But it is not completely satisfactory in middle scales, where occurs the
crossover between enthalpic and entropic behavior.\citet{hsu_breakdown_2011}

$$H_{bend}=\frac{\kappa}{2} \int ds|\frac{\partial \textbf{t}}{\partial s}|$$

where $k\equiv\kbend{}$ is the bending modulus, \textbf{t} is a unit tangent
vector along the chain and $s$ is the coordinate of the position along the
backbone. In the WLC model we can relate the persistance length with the bending
modulus via: $\lp{}=2\kbend{}/((d-1)\kbolt{}T)$ where $d$ is the space
dimensionality.

For nearly straight filaments $\frac{\partial \textbf{t}}{\partial s}$ can be
expressed via the transverse deviation $u(x)$ of the chain from its straight
conformation. $\frac{\partial \textbf{t}}{\partial s}=u''(x)$. If the chain is
under a tensional force from one end (and the other end fixed), we can add the
term $-fL$ to the Hamiltonian, where L being the end to end distance. The
Hamiltonian with this force $f_B$ in transverse coordinates :
\begin{equation}\label{WLC_H}
H=\frac{1}{2}\int_0^{L_c} dx\Big[\kappa|u''|^2 + f_B|u'|^2\Big]
\end{equation}

where $L_c$ is the contour length of the chain.
The applicability of\ref{WLC_H} must be questioned in some cases since it
neglects excluded volume (steric repulsion effects) between the
chain constituents completely.\citet{hsu_breakdown_2011}. For very stiff polymer
these excluded volume considerations could be safely ignored.

Such a chain can respond to transverse and also to longitudinal forces by either
bending or stretching/compressing. We can explore further the force-extension
(FE) relationship, decomposing u in Fourier series:
\begin{equation}\label{WLC_ufourier}
u(x)=\sum_q u_q \sin(qx)
\end{equation}
with the wave vector $q=n\pi/L_c$. We can rewrite \ref{WLC_H} as:
\begin{equation}\label{WLC_Hq}
H=\frac{1}{2}\int_0^{L_c} dx\Big[\kappa|u''|^2 + f_B|u'|^2\Big] =
\frac{L_c}{4}\sum_q (\kappa q^4 + f_Bq^2)u_q^2
\end{equation}

To calculate the equilibrium amplitude we shall use the equipartition theorem:
\begin{equation}\label{equipartition}
\Big\langle x_m\frac{\partial H}{\partial x_n} \Big\rangle= \delta_{mn}\kbolt{}T
\end{equation}

where H is the Hamiltonian or energy function and $x_n$ corresponds to the
% degrees of freedom in the phase space. Note than the system must be ergodic and in
thermal equilibrium to apply the equipartition theorem. A system is ergodic when
the ensemble average (mean over all the possible states) is equal to the time average (mean over all time
steps). Also the equipartition theorem does not hold when the
thermal energy $\kbolt{}T$ is smaller than the quantum energy spacing for a
particular degree of freedom because the breakage of the energy continuum. This
strong requirement does not hold for many soft matter systems. The idea behind
the equipartition theorem is that the energy in thermal equilibrium is shared
equally among all its components.

Applying \ref{equipartition}to \ref{WLC_Hq}, the equilibrium amplitudes
$u_q^{eq} $ satisfy:
\begin{equation}\label{equipartition_uq}
\langle u_q^{eq}\rangle=\frac{2\kbolt{}T}{L_c(\kappa{}q^4+f_Bq^2)}
\end{equation}
We can now calculate $\delta L$, the difference between the filament's contour
length and the equilibrium length.
\begin{equation}\label{deltaL}
\Delta L= \int dx \Big[ \sqrt{1+|\partial u/\partial x|^2} -1\Big]\simeq
\frac{1}{2} \int dx |\partial u/\partial x|^2 = L_c \sum_q q^2 u_q^2
\end{equation}

Using \ref{equipartition_uq} and including a factor of $2$ due to both
degrees of freedom of the transverse displacements from the straight
conformation:
\begin{equation}\label{deltaLmean}
\langle\Delta L\rangle = \kbolt{}T\sum_q \frac{1}{\kappa{}q^2+f_B}
\end{equation}

The result is most convenient expressed un terms of scaled difference between
the extension at force $f_B$ and that at zero force\citep{storm_nonlinear_2005}:
\begin{equation}\label{WLCdisplacement}
\delta l=\langle\Delta L\rangle_{f=0} - \langle\Delta L\rangle_{f_B} =
\frac{L_c^2}{\ell_p\pi^2} \sum_q \frac{\varphi}{n^2(n^2 + \varphi)}
\end{equation} 
where $\varphi = f_BL_c^2/\kappa{}\pi^2$

\ref{WLCdisplacement} can be inverted to yield a force-extension (FE)
relation:\citep{marko_stretching_1995}:
\begin{equation}\label{FEMarko}
f(x)=\frac{\kbolt{}T}{\ell_p} \Big[ \frac{1}{4(1-(x/L_c)^2)}
-\frac{1}{4}+\frac{x}{L_c} \Big]
\end{equation}

which diverges as $f \sim (x - L_c)^{-2}$ as $x\rightarrow L_c$
%TODO: ADD INFO ABOUT FE

This force-extension curve is of central importance in this project. This FE can
be measured using optical tweezers, experimental technique available in our
group. Optical tweezers are able to trap and manipulate beads of nanometric
size with high precision\cite{dasdsa}. Biopolymer chains can be attached to
these beads using linkage molecules, and then study the force-extension of the
chain when the force is applied by the movement of the trap in the tweezers, and
the extension is measured tracking the beads under the microscopy.
%  Since the bending rigidity of the polymer is
% of the thermal energy order, the equilibrium length L_{eq} is determined by the
% transverse fluctuations of u.



 Suppose both ends of a chain
are fixed in space. If the chain is rigid -a rod-, there is only one possible conformation
(if we assume there is rotational symmetry), if the chain is flexible, there
are a lot possible conformations or paths between the two ends. In the later
case, the system will be dominated by entropic effects.

A polymer chain can be characterized as an ideal chain, where there aren't
interaction between monomers separated by many bonds. In ideal chains the free
energy of the chain is completely entropic. The worm-like chain is an ideal
model valid for stiff polymers, are shows an excellent agreement with stiff
biopolymers as double stranded DNA.
 pg 88 Rubinstein:
R
P
f= entropic spring.

In a real chain, where correlation between monomers are included.
For example with the excluded volume term because other monomers is included.


Talk about Glassy worm like chain in networks


\gls{Lp}  
 (\citet{storm_nonlinear_2005})
\citet{stein_algorithm_2008}



\section{Worm like chain}
From botton up. WLC (chains) - Network level
\begin{enumerate}
  \item Freely joined .. Rubinstein
  \item WLC Rubinstein and Erich
  \item Experimental measure with optical traps.
\end{enumerate}

\section{Network}

\section{Rheology measurement - Modulus}
DWS, cylinders, optical traps. Full of methods.



%----------------------------------------------------------------------------------------
%	SECTION 2
%----------------------------------------------------------------------------------------

