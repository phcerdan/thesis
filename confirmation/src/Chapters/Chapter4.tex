% Chapter Template

\chapter{Schedule and future work} % Main
% chapter title

\label{Chapter-future} % Change X to a consecutive number; for
% referencing this chapter elsewhere, use \ref{ChapterX}

\lhead{Chapter 3. \emph{Schedule and future work}} % Change X to a consecutive
% number; this is for the header on each page - perhaps a shortened title

%----------------------------------------------------------------------------------------
%	SECTION 1
%----------------------------------------------------------------------------------------

\section{Schedule}

\begin{description}
  \item[0-6 months] Landing. Literature overview. Planification and research on
  computational tools.
  \item[6-12 months] Reconstruct-network software development.
   Access to image analysis tools. Avizo and Matlab.
  \item[12-18 months] Selection/development of proper model in order to simulate
  the strain of the network. Think about how simulate long time behavior in the
  networks.
  \item[18-24 months] Explore models and other theoretical frameworks.
  Scaling and renormalization group theory, graph theory to explore strain
  pathways and other networks properties. Plausible experimental experience if
  required.
  \item[24-30 months] Put all together with a solid bottom-up and up-bottom
   approach convergent in the mesoscale. Image analysis of real networks,
   dynamics simulation and in-silico reconstructed networks.
   \item[30-36 months] Conclusions and writing of the
   thesis report.
\end{description}

\subsection{Funding:}
This PhD project is funded by the MacDiarmid and Riddet
institute. 

%-----------------------------------
%	SUBSECTION 1
%-----------------------------------
\section{Future directions and interesting questions}

The field is full of opportunities and in continuous development. The exact
direction to follow will rely on further discussion with my supervisor, and the necessities
of the project from which I am receiving funding. The structure of that project,
and my role on it is clear.

\subsection{``Mesocule'' project}
The project is formed by $3$ professors working with $2 Post-docs$
and supervising $2 PhD students$. A post-doc in Christchurch is able to build We
have a meeting every $6$ months to report advances and choose new directions. A group
member is able to measure with optical tweezers the force-extension curve of
fibrils. Then, a bottom-up approach can be
followed\citep{schuster_investigating_2012} to model the mechanical properties of the
 whole networks from the single-chain properties. My developed software about
 reconstructing network will permit us to study how the architecture of the
 network affect the bulk properties of the
 material.


%-----------------------------------
%	SUBSECTION 2
%-----------------------------------

\subsection{Long time behaviour: network quakes and aging}
There are reports in the literature \citep{kajiya_slow_2013}, and also by group
mates, that at long times the network is affected by sudden de-correlations over
a few tenths of a second occur. The more plausible conjecture is that such quakes are associated with the release of chain constraint due to unbinding of
cross-links. This is reported experimentally in our group with micro-rheology
DWS techniques in the lab. Current model of networks as the Glassy WLC model,
cannot predict this behaviour. A computer simulation may shed some light on the
topic.

Aging, i.e. slow change of the physical properties over time, and non-ergodicity
i.e. the time average of a succession of events is not the same as the
ensemble average. This behaviour is related to glassy systems, where long
orders correlations(solids) no longer exists, but there is still local order and
homogeneities.\citep{cipelletti_slow_????}

\subsection{Cluster of nodes as mesoscale objects}
The resolution of image analysis is not enough to provide the network
architecture in some high density areas of the material. These clusters could
then be treated as new mesocule object, similar to nodes, but with different
properties to those that represent cross-links. This scaling approach could be
study with the renormalization group theory, which has been successful to explain
scaling behavior of polymers in the past.\citep{gennes_scaling_1979}

\subsection{Computational methods}  
The last Nobel prize in Chemistry was given "for the development of multi-scale
models for complex chemical systems" \citep{nobel:chemistry2013} These methods
allow the simulation of protein structures,  docking of drugs in cells and
multiple complex chemical reactions.

These coarse-graining methods study really complex
systems,\citep{de_pablo_coarse-grained_2011} impossible to compute with  the
fine detail potentials of molecular dynamics, but with effective potentials 
derived from scaling, coarse grain and renormalization group theories. These
effective potentials contain the information of the lower scales and can be
tweaked if that information changes, but it is not necessary to simulate all the
degrees of freedom of the fine structure. The communication over scales is the
responsible of the success of the method, where changes in the lower scales are
highly correlated with scales larger by several orders of magnitude. To achieve
this advanced Monte Carlo methods are in continuous development.
\citep{karayiannis_novel_2002}

Other interesting and widely used way of simulation is finite element methods
(FEM). This method solves differential equations with boundary conditions,
connecting many simple element equation over many small subdomains, named finite
elements. This connexion approximate a more complex equation over a larger
domain.


% ---------------------------------------------------------------------------------------

% \section{Future work}

