% Chapter Template

\chapter{Reconstructing networks from statistical distributions} % Main
% chapter title

\label{Chapter-Reconstruction} % Change X to a consecutive number; for
% referencing this chapter elsewhere, use \ref{ChapterX}

\lhead{Chapter 3. \emph{Reconstructing the network from statistical
distributions}} % Change X to a consecutive number; this is for the header on each page - perhaps a shortened title

%----------------------------------------------------------------------------------------
%	SECTION 1
%----------------------------------------------------------------------------------------
\section{Brief intro}

\section{Methods}

The  software used has been independently developed here, but is based on the
idea of \citet{lindstrom_biopolymer_2010} about reconstructing the network
architecture of polymers networks from a statistical description. They were
based on \citet{yeong_reconstructing_1998,yeong_reconstructing_1998-1} work
about  reconstructing random porosity media.\\
The input for the reconstruction is the
statistical distributions of $3$ properties in this case. This is the minimal 
set to reconstruct the correct architecture of the network to study the
mechanical properties, but other parameters could be added to study other or
more complex situations.  For example, if the structure of the edges / chains is
heterogeneous  (bio-functional areas, or different charge distribution), extra
statistical distributions can be added with minimal effort.

So in order to regenerate the network in-silico, the following distributions
should be obtained from the image analysis:
\begin{figure}[h!]
\begin{center}
\includegraphics[width=1.0\textwidth]{Figures/chapter-reconstruct/lindstrom-paper-images.png}

\caption[Collagen distributions]{Collagen I distribution measured by Lindstrom
\citep{lindstrom_biopolymer_2010}}

\end{center}
\end{figure}

\begin{enumerate} 
\item \textbf{Degree of nodes $N(p)$.} i.e. number of edges per node
or cross-link.
\item \textbf{Length of edges $P(\ell$).} Length of the edges
between connected
% nodes.
There are two options to define this length.
\begin{itemize}
\item End to end vector distance (EtE-length). Consider the edge as a straight
line between nodes.

\item Contour length (curved). Take into account all the path of the chain,
usually curved.
\end{itemize}

The difference between both is less that one
might have expected at a first glance. \citet{nisslert_identification_2007},
who use a different reconstructing algorithm, report that straight and curved
lengths have similar distribution, although the straight lines were about a
$15\%$ shorter.\\
I expect that high flexibility of the chains and/or low density of cross-links
will increase the differences between them. Biopolymer are semi-flexible, stiff
enough to keep this difference low.
In any case, if both cases are following the same distribution is just a matter
of scaling the length parameters.\\
The difference between these lengths can be used as a measure of the
persistence length (\gls{Lp}) of the chains.
  

\item \textbf{Cosine director $B(\beta$)} between incident edges. Take into
account the relative orientation of the edges incident to a node.


\end{enumerate}

These parameters can be obtained using \textbf{image analysis} of data from
\gls{confocal}, transmission electron microscopy (\gls{TEM}) or
scanning electron microscopy \gls{SEM}. See chapter \ref{Chapter-Image} for
more information about this process.

The following distributions are used to develop the software as seen in
Lindstrom paper \citep{lindstrom_biopolymer_2010}. He chose these distributions
because they fit the data taken from image analysis of collagen I, but they make
no claim about their use in other systems. I have noticed differences comparing
 the collagen distribution proposed here with our data in actin networks. The
 direction cosines distribution has a lot of variability, and it seems to
 follow a different distribution, however the degree and length distribution
 have similar shape in both systems.


\begin{enumerate} 
\item \textbf{$N(p)$}. The node degree follow a shifted geometric distribution.

\begin{equation} \label{degree-dist}
N(p)=q(1-q)^{p-3} 
\end{equation}
where $q=1/(Z-2)$ , and Z is the average node degree.

The distribution is shifted because the minimum degree of a node is $p=3$.$p=1$
corresponds to dead-ends nodes, and $p=2$ correspond to bending nodes. We can ignore dead-ends because they don't
affect to the mechanical properties of the bulk network. And the bending nodes
can be ignored if its $2$ edges are merged into a unique, larger edge,
connecting the neighbors of this bending node between them. Some information is lost in this
process if we are characterizing the length of edges by end to end
vectors.

\item \textbf{$P(\ell)$} The length distribution was found to be
logarithmic-normal
\begin{equation} \label{length-dist}
P(\ell)=\frac{1}{\ell
s\sqrt{2\pi}}\exp{\bigg[-\frac{(\mu-\ln{\ell})^2}{2s^2}\bigg]}
\end{equation}
where $\mu$ and s are the mean and standard deviation of $\ln{\ell}$
respectively.

%CHECK THAT LINDSTROM SAYS THAT \ell is normalized by n^{-1/3}
\item \textbf{$B(\beta)$} The director cosines distribution could be represented
by a truncated power series in the case of collagen.
\begin{equation} \label{cosines-dist}
B(\beta)=\sum_{k=1}^{m} b_k(1-\beta)^{2k-1}
\end{equation}
If we truncate it at $m=3$, and $B(\beta)$ is normalized to unity, then there
are two parameters left, $b_1$ and $b_2$ (or other combination).
\end{enumerate}

These $3$ distributions are enough to fully characterize the architecture of a
network of collagen. So, the network can be reconstructed with five
independent parameters:
$\mu$, s, $Z$, $b_1$ and $b_2$.


The output of the algorithm is a network or spatial \gls{graph}, which is formed
by $2$ sets: a set of nodes, containing the nodes
position and the degree; and a second set of edges, which connect pair of
nodes. The software
does not allow parallel edges, i.e. two edges connected the same pair of nodes,
and the network is fully connected,  there are no nodes or clusters of nodes
isolated  from the rest of the network. Not allowing parallel edges could be
unrealistic in some situations, to deal with that, edges could be characterized
by a thickness parameter, this parameter could be associated to the number of
parallel edges in that edges.
\section{Euclidean Graph Generation Algorithm}
The Euclidean graph generation (EGG) algorithm is used to reconstruct a $3D$
network on a periodic cube domain $\Sigma$ of size $\Lambda$. The algorithm uses
a Monte Carlo method with Simulated annealing, this is a general heuristic
method for global optimization which avoid the convergence to local minimums on
the energy landscape.

The following steps describe the algorithm:
\begin{enumerate}[label=\textbf{\Roman*}]
  \item A \textbf{initial graph configuration $H_0$} is generated by placing
  nodes drawn from a randomly uniform distribution inside the domain $\Sigma$. Then a
  valency drawn from the degree distribution $N(p)$ is assigned to each node.
  And finally, pairs of nodes are connected by edges, but not exceeding the
  fixed degree of any node. The initial configuration does not follow neither
  the length not the director cosine distribution.
  
  In practice, I have used the complex systems library Igraph to generate $H_0$
  with the described constraints, see the attached code documentation for more
  details.
  \item There are two possible movements to \textbf{update} from a graph $H$ to
  $H'$.
    \begin{enumerate}[label=\textbf{\alph*)}]
    \item Remove at random two edges from $H$ and then add other two edges
    without changing the valence of the nodes. This hard constraint only allows
    $2$ possible edge configuration.
    \item Move the position of a node a random distance in the range $[0,\rho]$,
    where $\rho$ can be tuned to enhance convergence.
    
    Note that the valence of the nodes is unchanged by any of these updates.
  \end{enumerate} 
  
  \item We define a non-negative \textbf{``energy'' function} in a graph:
  $E(H)=A_P(H) + A_B(H)$, where $A_P$ and $A_B$ are the Cramer-von Mises test
  statistics for the distributions P and B respectively. The Cramer-von Mises
  test \citep{anderson_distribution_1962} compares a set of variables $x_1<x2<\ldots<x_n$ with a
  distribution $f$ and produces an associated value $A_f$. The smaller $A_f$ the better the set
  of data $\{x\}$ fit the distribution $f$.
  
  With this definition of energy, $E(H)$ has a global minimum for graphs with
  the target length,$P(\ell)$ and direction cosine $B(\beta)$ distributions.
  
  \item To find the graph with minimal $E(H)$ we use a \textbf{simulated
  annealing} algorithm. Starting at $H=H_0$, we attempt to update the graph to
  $H'$ with the equally probable transitions \textbf{a)} or \textbf{b)}.  If the
  energy of the new graph is lower than before $E(H')\leq E(H)$ the transition
  is accepted. If the energy of the new graph is greater, $E(H')> E(H)$, the
  transition still can be accepted with a probability $\exp\{[E(H)-E(H')]/T\}$. Here $T$ is an analogous to temperature, and it is
  the main characteristic of simulated annealing algorithms. This temperature
  decays exponentially with the number of accepted transitions, which reduces
  progressively the number of ``unfavorable" accepted transitions, but avoid the
  network to get stuck in local energy minimums.
  \item \textbf{Convergence} criteria. We will stop the algorithm, assuming
  that we have already converged to the desired graph, when the energy $E(H)$ is
  close to zero or lesser than a threshold.
  The threshold will depend on the number of nodes (size) of the network.
\end{enumerate} 

\section{Results}
\subsection{Testing in collagen I}
The code engine has been successfully tested, and it is able to reconstruct
generate a network from the $3$ statistical distributions. To test we have used
the collagen distributions that \citet{lindstrom_biopolymer_2010} measured in
the collagen data. In Fig.\ref{fig:collagen-network}\subref{collagen_param} we
can see a reduction of $99.9\%$ from the initial graph energy $E(H_0)$ to the
final graph, that as we can see in Fig.
\ref{fig:collagen-distributions} successfully follow the target distributions. 
Note that the
simulated and target-distributions are superimposed with each other, no fit is involved. 

\begin{figure}[h!]

\begin{minipage}{0.5\textwidth}
% \begin{minipage}{0.5\textwidth}
\begin{center}
\subfloat[]{%
\includegraphics[width=1\textwidth]{Figures/chapter-reconstruct/networkN10000.png}%
\label{collagen_network}}
\end{center}
\end{minipage}
\begin{minipage}{0.5\textwidth}
\begin{center}
% \begin{minipage}{0.5\textwidth}
\subfloat[]{%
\includegraphics[width=1\textwidth]{Figures/chapter-reconstruct/parametersN10000.png}%
\label{collagen_param}}
\end{center}
\end{minipage}

\caption[Testing with theoretical collagen data]{ Reconstructed collagen
network \subref{collagen_network} Visualization of reconstructed collagen
network using Matlab. \subref{collagen_param} Simulation parameters from the reconstruction
algorithm.}
\label{fig:collagen-network}
\end{figure}



\begin{figure}[h!]

\begin{minipage}{0.32\textwidth}
% \begin{minipage}{0.5\textwidth}
\begin{center}
\subfloat[]{%
\includegraphics[width=1\textwidth]{Figures/chapter-reconstruct/lengthN10000.png}%
\label{collagen_length}}
\end{center}
\end{minipage}
\begin{minipage}{0.32\textwidth}
\begin{center}
% \begin{minipage}{0.5\textwidth}
\subfloat[]{%
\includegraphics[width=1\textwidth]{Figures/chapter-reconstruct/degreeN10000.png}%
\label{collagen_degree}}
\end{center}
\end{minipage}
\begin{minipage}{0.32\textwidth}
\begin{center}
\subfloat[]{%
\includegraphics[width=1\textwidth]{Figures/chapter-reconstruct/cosinesN10000.png}%
\label{collagen_cosines}}
\end{center}
\end{minipage}

\caption[Collagen: comparing target and simulated distributions]{ Comparison
between target distributions and simulated results in the reconstructed collagen
network.
\subref{collagen_length} Length Distribution $P(\ell)$.
\subref{collagen_degree} Degree distribution $N(p)$.
\subref{collagen_cosines} Director cosines distribution $B(\beta)$.
}
\label{fig:collagen-distributions}
\end{figure}

\subsection{Actin data}
The real use of the algorithm will be to reconstruct networks with different
statistical distribution. These statistics will be gathered from image analysis,
but also we can input any statistical distribution ``engineered'' by us, to
study the generated network architecture.

The image analysis of our data,(Chapter.\ref{Chapter-Image}) generates the
needed input for the reconstruction algorithm, i.e. the statistical
distributions of length, degree, and director cosines.
The length and degree distributions of actin data seems to follow a log-normal
and geometrical distribution respectively. These are the same distributions
proposed by Lindstrom for collagen (changing the parameters). But actin seems to
follow a different distribution for cosine directors, (see
Fig.\ref{fig:avizo_histograms30}) at least with the image analysis resulted from
Avizo.  I am currently working on a way to extract the cosines distributions from FIRE algorithm.  

 When we will have the parameters of the cosines distributions of actin, we
can easily implement it in the reconstruct software, and then generate
the network following those distributions.

The step in the future is to study the relation of the network architecture to
mechanical properties and dynamics of the network.

\begin{figure}[h!]
  \begin{center}
    \begin{tikzpicture}[scale=0.8]
      \node[blockimage] (A) at (2,8.5) {
      \includegraphics[width=1.0\textwidth]{Figures/chapter-image/fire/fire017.png}
      \textsc{Image Analysis}};
      \node[blocknetwork] (B) at (8.5,3) {%
%         \begin{minipage}{0.5\textwidth}
        \begin{center}
        \includegraphics[width=1.0\textwidth]{Figures/chapter-image/avizo/ActinAllan11jun132_BIN12LEN5.png}%
        \end{center}
%         \end{minipage}
%         \begin{minipage}{0.5\textwidth}
%         \begin{center}
%         \includegraphics[width=1.0\textwidth]{Figures/chapter-image/avizo/ActinAllan11jun132_BIN18LEN5.png}%
%         \end{center}
%         \end{minipage}
        
       \textsc{Network /}\\
       \textsc{Spatial Graph}
        };

      \node[blockimage] (C) at (1.5,0) {
      \textsc{Reconstruction}\\
      \textsc{Algorithm}
      };
      \node[blocknetwork] (F) at (0,3.5) {
      \includegraphics[width=1.0\textwidth]{Figures/chapter-reconstruct/lindstrom-paper-images.png}%
      
      \textsc{Statistics}
      };
%       \node (M) at (2,11) {X};

      \node[blockmodel] (D) at (15,3) {\textsc{Modeling}}; 

%       \draw[pink,->] (A) -- (B);
%       \draw[pink,->] (B) -- (Q);
%       \draw[pink,->] (G) -- (M); 
      \draw[green,arrows={-triangle 45}] (A) -- (B);
      \draw[blue,arrows={-triangle 45}] (B) -- (F);
      \draw[blue,arrows={-triangle 45}] (F) -- (C);
      \draw[red,arrows={-triangle 45}] (B) -- (D);
      \draw[green,arrows={-triangle 45}] (C) -- (B);

%       \draw[green,->] (I) -- (S);
%       \draw[blue,->] (H) -- (R);
%       \draw[red,->] (A) -- (B);
%       \draw[red,->] (E) -- (T);
    \end{tikzpicture}
  \end{center}
  \caption[Scheme of the thesis]{Scheme of the project. }
\end{figure}

