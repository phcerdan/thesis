% Chapter Template

\chapter{Reconstructing networks from statistical distributions} % Main
% chapter title

\label{Chapter-Reconstruction} % Change X to a consecutive number; for
% referencing this chapter elsewhere, use \ref{ChapterX}

\lhead{Chapter 3. \emph{Reconstructing the network from statistical
distributions}} % Change X to a consecutive number; this is for the header on each page - perhaps a shortened title

%----------------------------------------------------------------------------------------
%	SECTION 1
%----------------------------------------------------------------------------------------
\section{Brief intro}

\section{Methods}

The  software is based, but independently developed,  on the idea of
\citet{lindstrom_biopolymer_2010} about reconstructing the network
architecture of polymers networks from a statistical description. They were
based on \citet{yeong_reconstructing_1998,yeong_reconstructing_1998-1} work
about  reconstructing random porosity media.\\
The input for the reconstruction is the
statistical distributions of $3$ properties in this case. This is the minimal 
set to reconstruct the correct architecture of the network to study the
mechanical properties, but other parameters could be added to study other or
more complex situations.  For example, if the structure of the edges / chains is
heterogeneous  (bio-functional areas, or different charge distribution), extra
statistical distributions can be added with minimal effort.
\begin{enumerate} 
\item \emph{N}. Degree of nodes. i.e. number of edges per node or cross-link. 
\item \emph{$\ell$}. Length of edges. Length of the edges between connected
% nodes.
There are two options to define this length.
\begin{itemize}
\item End to end vector distance (EtE-length). Consider the edge as a straight
line between nodes.

\item Contour length (curved). Take into account all the path of the chain,
usually curved.
\end{itemize}

The difference between both is lesser that one
might have expected at a first glance. \citet{nisslert_identification_2007},
who use a different reconstructing algorithm, report that straight and curved
lengths have similar distribution, although the straight lines were about a
$15\%$ shorter.\\
I expect that high flexibility of the chains and/or low density of cross-links
will increase the differences between them. Biopolymer are semi-flexible, stiff
enough to keep this difference low.
In any case, if both cases are following the same distribution is just a matter
of scaling the length parameters.\\
The difference between these lengths can be used as a measure of the
persistence length (\gls{Lp}) of the chains.
  

\item \emph{$\beta$}. Cosine director between incident edges. Take into account
the relative orientation of the edges incident to a node.


\end{enumerate}

These parameters can be obtained using \textbf{image analysis} of data from
\gls{confocal}, transmission electron microscopy (\gls{TEM}) or
scanning electron microscopy \gls{SEM}. See chapter \ref{Chapter-Image} for
more information about this process.

The follow distributions are used because they fit the data taken from image
analysis, but I make no claim about their use in other systems. The
direction cosines distribution has a lot of variability. I have experienced
these differences between actin and collagen data.
\begin{enumerate} 
\item The node degrees follow a shifted geometric distribution. Shifted because
the minimum degree of a node is $p=3$.$p=1$ corresponds to dead-ends nodes, and
$p=2$ correspond to bending nodes. We can ignore dead-ends because they don't
affect to the mechanical properties of the bulk network. And the bending nodes
can be ignored if its $2$ edges are merged into a unique, larger edge,
connecting the neighbors of this bending node between them. Some information is lost in this
process if we are characterizing the length of edges by end to end
vectors.
\begin{equation} \label{degree-dist}
N(p)=q(1-q)^{p-3} 
\end{equation}
where $q=1/(Z-2)$ , and Z is the average node degree.
\item The length distribution is logarithmic-normal  
\begin{equation} \label{length-dist}
P(\ell)=\frac{1}{\ell
s\sqrt{2\pi}}\exp{\bigg[-\frac{(\mu-\ln{\ell})^2}{2s^2}\bigg]}
\end{equation}
where $\mu$ and s are the mean and standard deviation of $\ln{\ell}$
respectively.
%CHECK THAT LINDSTROM SAYS THAT \ell is normalized by n^{-1/3}
\item The director cosines distribution can be represented by a truncated power
series.
\begin{equation} \label{cosines-dist}
B(\beta)=\sum_{k=1}^{m} b_k(1-\beta)^{2k-1}
\end{equation}
If we truncate it at $m=3$, and $B(\beta)$ is normalized to unity, then there
are two parameters left, $b_1$ and $b_2$ (or other combination).
\end{enumerate}

Using these $3$ distributions, the network can be reconstructed with five
independent parameters: $\mu$, s, $Z$, $b_1$ and $b_2$.

These parameters are enough to fully characterize the architecture of a
network if you are interested in computing the
mechanical properties of the system.

The output of the algorithm is spatial \gls{graph}, which is formed by $2$
sets. The first set is associated with the nodes, and contains the data of node
position and degree. The second set is a list of edges, an edge is defined
by two nodes which the edge is connecting. The software does not allow parallel
edges, i.e. two edges connected the same pair of nodes, and the network is fully
connected, there are no nodes or clusters of nodes isolated from the rest of the
network.

\section{results}
Images, comparison between theoretical distributions and simulated measures.
Parameter screenshot.

\section{Possible improvements of code}
octree or kd tree, partition of space. Problems then with montecarlo simulation.
It is not a random pick up of nodes, if we discard neighbor nodes that are
pretty far away from the first selected node.

\section{Merge with image analysis}
\subsection{Actin vs collagen: required changes in the cosine director
distribution to fit image analysis results}
