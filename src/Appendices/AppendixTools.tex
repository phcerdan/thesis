% Appendix A

\chapter{Acquired tools} % Main appendix title

\label{Appendix-tools} % For referencing this appendix elsewhere, use
% \ref{AppendixA}

\lhead{Acquired tools} % This is for the header on each page -
% perhaps a shortened title

\section{Coding : C++}
At the beginning of my PhD, I was not sure about what language to choose for the
development of tools to compute in-silico polymer networks. I had previous
experience with Fortran and Python, but each of them has some drawbacks, and I wanted to start
the project with the most sharp tools available. From my point of
view, the more important characteristics of a language are three :
\begin{enumerate*}[label=\bfseries\alph*)]
\item Fast computation. 
\item Community support and development tools.
\item Learning curve.
\end{enumerate*}
 
Fortran is really fast, its vector/matrix multiplication is unbeatable. The main
drawback of Fortran is that is hard to develop complex projects on it(it easy
to end with a unstructured/messy code), no Object Oriented Programming (OOP)
support and also it has no use at all in any other field different from hardcore numerical computing. Also the community is
tiny, and the language lacks of some really useful libraries.
Even though it is still used in many big projects, sometimes because the project has inherited old code,
 or sometimes because the Fortran speed is needed. 

Python has a huge community, it is a pretty modern language, easy development,
OOP support, widely used in a lot of fields. Programming in Python is kind of a
pleasure because it is closer to the regular way of thinking than any other
language. The unique drawback is that it is really slow, by orders of magnitude
in comparison to Fortran or C++. 

So, I chose to learn C++. I have access to tons of libraries (I am using
Boost Graph Library -to generate the spatial graph-, and Igraph -complex
networks- in my Reconstruct-Network software), it is also the language of $3D$  computing
graphics, in case I want to visualize the networks at fast speed.  It is almost
as fast as Fortran (even faster in some points thanks to the continuous
development of compilers). The drawback of C++ is that is hard to enter and
start learning, but with help of
\href{http://www.stackoverflow.com}{StackOverflow},
\citet{stroustrup_c++_2013}, and \citet{lippman_c++_2013}, I have developed the
Reconstruct-Network software (Chapter \ref{Chapter-Reconstruction}),
and I think I have gained enough expertise in C++ to face future challenges in this project.

\emph{``Premature optimization is the root of all evil.'' Donald Knuth}
\section{Linux environment}

I have gained expertise on Linux
platform, the open source community, and also with some useful software that I
use daily.
 
 \begin{itemize}
   \item Eclipse IDE. Graphic interface for different languages. I have used
   it to work with C++, Python, and this latex manuscript.
   \item Git. Version control software. Software to keep controlled all
   changes in the software under development. Version control is vital to handle
   complex software, and really useful to explore,add or remove modules  with no
   risk of mess it up all previous work. It is also fundamental in collaboration
   projects with other programmers. \href{https://github.com/}{GitHub} is really the hub
   of amazing open source projects created and maintained from the open source community.
   \item Cmake. Multiplatform assistant for managing the build process of
   software. It is required for using computer graphic software, such VTK.
   \item \href{http://www.zotero.org/}{Zotero}. It is a multiplatform open
   source tool for bibliography management (alternative to EndNote, or
   Mendelev).It can be used as an add-on to any internet browser (Firefox -open source- is recommended), and also has a 
   stand-alone version.
 

 \end{itemize}
 





