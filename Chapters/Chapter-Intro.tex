% Chapter Template

\chapter{Introduction} % Main chapter title

\label{Introduction} % Change X to a consecutive number; for referencing
% this chapter elsewhere, use \ref{ChapterX}
%Default was: Chapter 1 .\emph{Introduction} 
\lhead{\emph{Introduction}} % Change X to a consecutive number;
% this is for the header on each page - perhaps a shortened title

\section{Soft matter and biopolymers}
Soft matter, also known as complex fluids, is a subfield of condensed
matter, comprising systems which organize at many different length scales, into
many forms and classify as something in between the
ordered solids and disordered liquids. Its conformation is heavily
influenced by thermal fluctuations in the energy scale comparable to the room
temperature. This characteristic allows conformational changes where complex
behaviour can occur, life as the outermost example.


Soft matter includes liquids, colloids, polymers, foams, gels, granular
materials, surfactants, liquids crystals and some biological materials.
Pierre-Gilles de Gennes, who has been called the ``founding father of soft
mater'' received the Nobel Prize in physics in $1991$ ``for discovering that
methods developed for studying order phenomena in simple systems can be
generalized to more complex forms of matter, in particular to liquid  crystals
and polymers''\citep{de_gennes_pierre-gilles_????}

The biological soft materials span as well different length scales: from sugar
chain structures to active gels inside cells. In the
lowest scale, we find the biopolymers, which are classified as polysaccharides
(cellulose, pectin),  polynucleotides (RNA, DNA) or polypeptides (proteins).

Most biopolymers are considered semi-flexible, something in between
rigid rods and completely flexible (loose) chains.  To formalize this
distinction we need to introduce $2$ parameters that characterize a polymer
chain: the \emph{persistence length} \gls{Lp} which is the typical length scale
for the decay of tangent-tangent correlations, and the \emph{contour length}
\gls{Lc}, which is defined as the maximum end-to-end distance of a linear
polymer chain.

A chain is considered flexible when $\ell_p<<L_c$, and rigid when the opposite
holds. Completely flexible chains exhibit a purely entropic elastic
response, and rigid filaments display no entropic, but purely enthalpic
response. Semi-flexible biopolymers have a similar magnitude of $\ell_p$ and
$L_c$. This kind of filaments do not form loops or knots, but they are
flexible enough to have thermal bending
fluctuations\citep{storm_nonlinear_2005}. They behave like rods at scales
smaller than $\ell_p$ and like random coils in larger scales where the entropic behavior
dominates.

\section{Semiflexible single chains: WLC model}
Single semiflexible chains can be modeled with great success with the
\emph{worm-like chain} (WLC) also known as \emph{Kratky-Porod} model.
\citep{rubinstein_polymer_2003, schuster_hierarchical_2011}. This model
describes the single chain as an idealized curve that resists bending. Such
bending deformations are described by the following Hamiltonian:
$$H_{bend}=\frac{\kappa}{2} \int ds|\frac{\partial \textbf{t}}{\partial s}|$$

where $k\equiv\kbend{}$ is the bending modulus, \textbf{t} is a unit tangent
vector along the chain and $s$ is the coordinate of the position along the
backbone. In the WLC model we can relate the persistance length with the bending
modulus via: $\lp{}=2\kbend{}/((d-1)\kbolt{}T)$ where $d$ is the space
dimensionality.

For nearly straight filaments $\frac{\partial \textbf{t}}{\partial s}$ can be
expressed via the transverse deviation $u(x)$ of the chain from its straight
conformation. $\frac{\partial \textbf{t}}{\partial s}=u''(x)$. If the chain is
under a tensional force from one end (and the other end fixed), we can add the
term $H_{external}=-fL$ to the Hamiltonian, where L being the end to end
distance.
The Hamiltonian with this force $f_B$ in transverse coordinates is
\begin{equation}\label{WLC_H}
H=\frac{1}{2}\int_0^{L_c} dx\Big[\kappa|u''|^2 + f_B|u'|^2\Big]
\end{equation}

where $L_c$ is the contour length of the chain.
The applicability of equation \ref{WLC_H} must be questioned in some cases since
it neglects excluded volume (steric repulsion effects) between the
chain constituents completely \citep{hsu_breakdown_2011}. However, for very stiff
polymer these excluded volume considerations could be safely ignored.

Such a chain can respond to transverse and also to longitudinal forces by either
bending or stretching/compressing.  We can explore further the force-extension
(FE) relationship, decomposing u in Fourier series:
\begin{equation}\label{WLC_ufourier}
u(x)=\sum_q u_q \sin(qx)
\end{equation}
with the wave vector $q=n\pi/L_c$. We can rewrite \ref{WLC_H} as:
\begin{equation}\label{WLC_Hq}
H=\frac{1}{2}\int_0^{L_c} dx\Big[\kappa|u''|^2 + f_B|u'|^2\Big] =
\frac{L_c}{4}\sum_q (\kappa q^4 + f_Bq^2)u_q^2
\end{equation}

To calculate the equilibrium amplitude we shall use the equipartition theorem:
\begin{equation}\label{equipartition}
\Big\langle x_m\frac{\partial H}{\partial x_n} \Big\rangle= \delta_{mn}\kbolt{}T
\end{equation}

where H is the Hamiltonian or energy function and $x_n$ corresponds to the
% degrees of freedom in the phase space. Note than the system must be ergodic and in
thermal equilibrium for an ergodic system. A system is ergodic when
the ensemble average (mean over all the possible states) is equal to the time average (mean over all time
steps). Also, the equipartition theorem does not hold when the
thermal energy $\kbolt{}T$ is smaller than the quantum energy spacing for a
particular degree of freedom because the breakage of the energy continuum. This
strong requirement does not hold for many soft matter systems. The idea behind
the equipartition theorem is that the energy in thermal equilibrium is shared
equally among all its degrees of freedom.

Applying equation \ref{equipartition} to \ref{WLC_Hq}, the equilibrium
amplitudes of the different Fourier modes $u_q^{eq} $ satisfy:
\begin{equation}\label{equipartition_uq}
\langle u_q^{eq}\rangle=\frac{2\kbolt{}T}{L_c(\kappa{}q^4+f_Bq^2)}
\end{equation}
We can now calculate $\delta L$, the difference between the filament's contour
length and the equilibrium length.
\begin{equation}\label{deltaL}
\Delta L= \int dx \Big[ \sqrt{1+|\partial u/\partial x|^2} -1\Big]\simeq
\frac{1}{2} \int dx |\partial u/\partial x|^2 = L_c \sum_q q^2 u_q^2
\end{equation}

Using equation \ref{equipartition_uq} and including a factor of $2$ due to both
degrees of freedom of the transverse displacements from the straight
conformation:
\begin{equation}\label{deltaLmean}
\langle\Delta L\rangle = \kbolt{}T\sum_q \frac{1}{\kappa{}q^2+f_B}
\end{equation}

The result is most conveniently expressed in terms of scaled difference between
the extension at force $f_B$ and that at zero force\citep{storm_nonlinear_2005}:
\begin{equation}\label{WLCdisplacement}
\delta l=\langle\Delta L\rangle_{f=0} - \langle\Delta L\rangle_{f_B} =
\frac{L_c^2}{\ell_p\pi^2} \sum_q \frac{\varphi}{n^2(n^2 + \varphi)}
\end{equation} 
where $\varphi = f_BL_c^2/\kappa{}\pi^2$

Equation \ref{WLCdisplacement} can be inverted to yield a force-extension (FE)
relation:\citep{marko_stretching_1995}:
\begin{equation}\label{FEMarko}
f(x)=\frac{\kbolt{}T}{\ell_p} \Big[ \frac{1}{4(1 - x/L_c)^2} 
-\frac{1}{4}+\frac{x}{L_c} \Big]
\end{equation}

where the first term in brackets dominates at the
high force regime, and the last two terms $-\frac{1}{4}+\frac{x}{L_c}$ are added
after the inversion of equation \ref{WLCdisplacement}, to fit the linear
elasticity observed experimentally at low force regime
\citep{marko_stretching_1995}.
This equation diverges as $f \sim (x - L_c)^{-2}$ as the end to end
distance approaches the contour length: $x\rightarrow L_c$.

This force-response, as in the case of rubber elasticity  for flexible
chains, is dominated by entropy at high forces. Because there are many
bend configurations and only one that is perfectly straight, stretching the
chain reduces its conformational entropy and thus produces an opposing
force.
%TODO: ADD INFO ABOUT FE

This force-extension curve of a single-chain is of central importance for homogeneous,
and isotropic networks with affine regimes. The mechanical response of a network with these the mentioned characteristics can be derived from a single-chain force-extension curve \cite{storm_nonlinear_2005}.

This FE can be measured using optical tweezers or \gls{AFM}.
Optical tweezers are able to trap and manipulate beads of
micron size with high
precision\citep{svoboda_direct_1993,svoboda_biological_1994}.
Biopolymer chains can be attached to these beads using linkage molecules, and then the force-extension of the
chain when the force is applied can be studied by the movement of the trap in
the tweezers.
The extension is measured tracking the beads using microscopy.

\begin{figure}[ht]
  \begin{subfigure}{0.65\textwidth}
    \includegraphics[width=0.9\textwidth]{Figures/chapter-intro/tweezers_configuration2.png}%
    \caption{\label{tweezers-configuration}}
  \end{subfigure}
  \begin{minipage}{0.35\textwidth}
  \begin{subfigure}{0.99\textwidth}
    \includegraphics[width=0.9\textwidth]{Figures/chapter-intro/tweezers_particles.png}%
    \caption{\label{tweezers-particles}}
  \end{subfigure}%

  \begin{subfigure}{0.99\textwidth}
    \includegraphics[width=0.9\textwidth,height=3cm]{Figures/chapter-intro/tweezers_linkage.png}%
    \caption{\label{tweezers-linkage}}
  \end{subfigure}
\end{minipage}

\caption[Optical Tweezers]{\protect\subref{tweezers-configuration} Current
optical tweezers configuration in our lab. \protect\subref{tweezers-particles}
Exertion of force moving the right trap. One of the bead (red-left) is fixed by
the most powerful trap. \protect\subref{tweezers-linkage} Linkage of the
 polymer stained with fluorescence particles (double stranded DNA) to the beads,
 and below, visualization of the fluorescence. Images by courtesy of Sandy Suei. }
\label{fig:optical_tweezers}
\end{figure}

%  Since the bending rigidity of the polymer is
% of the thermal energy order, the equilibrium length L_{eq} is determined by the
% transverse fluctuations of u.

Thanks to optical tweezers and Atomic Force Microscopy
(AFM)\citep{janshoff_force_2000} force-extension curves of many
biopolymers have been measured, e.g. DNA\citep{marko_stretching_1995},
polysaccharides\citep{marszalek_atomic_1999} and others. At low strain, the WLC
model captures well the system behavior, but at higher forces, a hookean
extension regime must be incorporated to account for enthalpic stretching
beyond the contour length.
Such a model is called \emph{extensible wormlike chain}
(EWLC)\citep{wang_stretching_1997}:

\begin{equation}\label{EWLC}
f(x)=\frac{\kbolt{}T}{\ell_p} \Big[ \frac{1}{4(1 - x/L_c + f(x)/K_0)^2}
-\frac{1}{4}+\frac{x}{L_c} + \frac{f(x)}{K_0}\Big]
\end{equation}


where $K_0$ is the elastic spring constant. Another model, the
\emph{clickable extensible wormlike chain} (CEWLC) incorporates extra
enthalpic components of the chain, for example taking into
account the conformational changes of sugar rings induced by high forces in polysaccharides\citep{haverkamp_model_2007}.


\begin{figure}[ht]
\begin{center}
\includegraphics[width=0.7\textwidth,height=0.5\textwidth]{Figures/chapter-intro/forceextension_CEWLC.png}%

\caption[Force extension curve: CEWLC]{Force-extension curve of a CEWLC model
from Ref. \citep{schuster_hierarchical_2011}. It fits the experimental
force-extension curve of a single polysaccharide (pectin) chain \citep{haverkamp_model_2007} }
\label{fig:force_extension_CEWLC}
\end{center}
\end{figure}

\section{Networks of semiflexible chains:}
After the study of single chains, we have to study how all those chains interact
with themselves, with each other and with the media where they are.
The WLC is a model for ideal chains because the interaction between
monomers of the chain are ignored. This is not a problem in straight
(i.e. stiff) chains, but would be unrealistic in coiled chains, which are better
modeled with a self avoiding walk, where distant monomers cannot occupy the
same position, or with the inclusion of an excluded volume, that takes into
account the interaction between monomers and the forbidden configurations.

The way the monomers of the chain interact with other monomers or with the
solvent has been studied for long time. The Flory interaction parameter $\chi$
which depends on temperature, and pressure of the system, is a way to
characterize it.
\begin{equation}\label{FloryInteraction}
\chi=\chi_{MS} - \frac{1}{2}(\chi_{MM} + \chi_{SS})
\end{equation}
where  $\chi_{MM}$ is related with monomer-monomer interactions. $\chi_{MS}$
with monomer-solvent, and $\chi_{SS}$ with solvent-solvent
interactions.

The relation between solvent, and monomers interaction can be classified using
$\chi$:
\begin{enumerate}
  \item Athermal. $\chi=0$. Solvent is very similar to other monomers.
  \item Good solvent. $\chi<1/2$. Monomers of the chain prefer to interact with
  the solvent than with other monomers.
  \item Poor solvent. $\chi>1/2$. Monomers prefer to be closer to other
  monomers. For example, hydrophobic materials in water solvent.
  \item Theta solvent. $\chi=1/2$. Occurs at a temperature $T=\Theta$, and
  corresponds to the exact cancellation between steric repulsion and van der
  Waals attraction between monomers. The excluded volume:
  \begin{equation}\label{excludedvolume}
  \upsilon=(1-2\chi)a^3
  \end{equation}
  where $a$ is the effective length between monomers, is zero in theta solvents,
  and chains behave as nearly ideal.\citep{gennes_scaling_1979}
\end{enumerate}

Also, the concentration of polymers has important effects on the system. The
overlap threshold parameter $c*$ acts as an order parameter to distinguish dilute
polymer solutions, where the coils are separate, and more concentrated solution
where the chains overlap. This threshold is not sharp, it is more properly
defined as a crossover between regions, but the scaling properties of $c*$ with
concentration are essential.
\begin{enumerate}
  \item Dilute. $c<c*$. Chains can be studied as an ideal gas, with almost no
  interaction between them.
  \item Dense. $c>c*$
  \item Semi-dilute. The chains do overlap but the polymer fraction
  $\phi=ca^3$ is still low $\phi*\ll\phi\ll1$. Since $\phi$ is small, the monomer-monomer interaction can
  be described very simply, such as with the excluded volume parameter
  $\upsilon$ in \ref{excludedvolume}. In the case of dense systems we would need more complex
  relations used in liquids/fluids systems.
\end{enumerate}


In dilute systems, polymer chains flow through
the solvent acting similar to a liquid state, in the other hand, dense system
are more like solids, but they are not completely frozen. The transition between
both regimes is called the gelation point. Th gelation is a connectivity
transition, where parts of the system becomes correlated by a path of
inter-connexion. In the gelation point, the chains stop to flow as a liquid. We
can think in an incipient gel, where a cluster of chains started to form (giant cluster), connecting the boundaries of the
material. This incipient gel will behave differently to a fully connected
network, where the giant cluster is connecting all the chains.

It is not easy to give a closed definition of
gels\citep{gennes_scaling_1979,rubinstein_polymer_2003}, I like to think about
it in these terms: it is the point in the phase space of parameters, where a significant correlation, between two points in the
spatial domain, appears over a length scale of the system size.
%Definition of gel, wikipedia UIAC.

\section{Bulk Rheology}
\subsection{Linear viscoelasticity}
Any elastic material can be studied as a linear spring, whereby the
extension (strain) $\gamma$ is proportional to the applied stress $\sigma$ (force/area) :
$\sigma=G'\gamma$ , where $G'$ is called the elastic modulus, and it is a
measure of sample's elasticity.

However, the material response to the force is not always that simple. Some
materials such as silk, rubber, and glass are subjected to a shear
stress and the corresponding Hooke's law deformation occurs, however it is
followed by an unexpected continuous deformation termed as ``creep''. Upon
removal of the shear stress, certain materials come back to the initial state,
while other are permanently deformed. The phenomena of a time dependant shear
response is called viscolelasticity. \citep{macosko_rheology:_1994}.

One way to study the deformation and flow of viscoelastic materials under
strain, is to externally apply a small amplitude
sinusoidal strain:
$\gamma(t)=\gamma_0\sin(\omega t)$. We can expect (not always, e.g. non-linear
regime) that the stress output of the material will be:
\begin{equation}\label{viscoelastic-response}
\sigma(t) = \sigma_0 \sin(\omega t + \delta)
\end{equation}
This stress output can be
analyzed by decomposition into two waves of frequency $\omega$, one in phase
with the strain, and the other $\pi/2$ out of phase:

\begin{equation}\label{viscoelastic-stress}
\sigma(t) = \sigma'(t) \sin(\omega t) + \sigma''(t)\cos(\omega t)
\end{equation}

where $ \tan \delta=\sigma''(t)/\sigma'(t)$. This allows the definition of two
moduli: $G'=\sigma'(t)/\gamma_0$ and $G''=\sigma''(t)/\gamma_0$ which are the
elastic (i.e. storage) modulus and viscous (i.e. loss) modulus respectively.

The stress can be expressed in terms of these two moduli:
\begin{equation}\label{viscoelastic-stress-G}
\sigma(t) = G'\gamma_0 \sin(\omega t) + G''\gamma_0\cos(\omega t) = G'\gamma(t)
 + \frac{G''}{w}\dot{\gamma}(t)
\end{equation}

$G''$ has a physical meaning, it is associated with the energy dissipation
per cycle of deformation, which provides and indication of viscous
loss \citep{macosko_rheology:_1994}. Note also that $\eta'=\frac{G''}{w}$
corresponds to the dynamic viscosity, used to characterize stress-response in pure liquids.

\subsection{Nonlinear behaviour: strain stiffening}
\label{intro-strainstiffening}
At small strain the stress-response is linear, but when the strain amplitude is
increased beyond a certain point, non-linear responses arise. In this situation,
the premise imposed in Eq.\ref{viscoelastic-response} that the stress response is linear does not
hold, however the qualitative study of $G'$ and $G''$ still provides information
about the nature of the non-linearities.
\citep{storm_nonlinear_2005,mackintosh_elasticity_1995,yao_nonlinear_2011,sheinman_nonlinear_2012,carrillo_nonlinear_2013}.

Fig. \ref{fig:strainstiffening-storm} shows that most of the biopolymer networks
share the same non-linear behaviour at high strains: strain-stiffening. In plain
words, the higher the strain is, the tougher the material becomes, or the more you pull,
the harder it is to pull more. The universal behavior in
biopolymers may have an evolutionary origin, due to the role of the strain-stiffening in biological functions such as cell motility
and mechano-transduction.

One way to study strain-stiffening experimentally is to use a \emph{large
amplitude oscillatory shear rheology} (LAOS) \citep{hyun_review_2011}. This
technique consists in pre-stress the network at a selected constant strain, and
then, study the frequency response applying other small amplitude sinusoidal strain, as in the
linear viscoelasticity situation.


 \begin{figure}[ht]
\begin{center}
\includegraphics[width=0.7\textwidth,height=0.5\textwidth]{Figures/chapter-intro/strainstiffening_storm.png}%

\caption[Strain-stiffening in semiflexible
polymers]{ Elastic shear moduli versus
strain from Ref.\citep{storm_nonlinear_2005}. Strain-stiffening behavior in
different cross-linked biopolymer networks
\citep{carrillo_nonlinear_2013}.}
\label{fig:strainstiffening-storm}
\end{center}
\end{figure}


The research on strain-stiffening in biopolymers is really
active \citep{storm_nonlinear_2005, onck_alternative_2005,
head_distinct_2003,head_deformation_2003,wilhelm_elasticity_2003,huisman_three-dimensional_2007,yao_nonlinear_2011}.
Stiffening can result from the response of the chain between cross-links, from
alterations in the network structure, or from both.

 In a seminal paper,
\citet{storm_nonlinear_2005} showed that: for low to middle strains, under
the assumptions that networks are homogeneous, isotropic, and strain uniformly
(affine deformations), the strain-stiffening is primarily due to the
longitudinal stiffening of the single chains (i.e.
the non-linearity of the force response curve such as
Fig.\ref{fig:force_extension_CEWLC}).

 \begin{figure}[ht]
\begin{center}
\includegraphics[width=0.6\textwidth]{Figures/chapter-intro/nonaffine.png}%

\caption[Affine and non-affine deformations]{ Difference between affine
and non-affine deformation of a network under shear. Figure adopted from
references \citep{wen_non-affine_2012,basu_nonaffine_2011}
}
\label{fig:nonaffine}
\end{center}
\end{figure}
The affine deformation assumption (Fig.\ref{fig:nonaffine}) is well-known in
network models for rubber elasticity, and allows for a relatively
simple description of the overall network response on the basis of the behavior
of a single filament. Predicting the viscoelastic response of networks under the affine assumption can be considered straightforward from current literature \cite{rizzi_importance_2016}.
But, this assumption does not hold for many systems.

The small-strain affine deformation assumption in two-dimensional
networks of straight filaments has
been studied in great detail by \citet{head_distinct_2003,head_mechanical_2005}
and \citet{wilhelm_elasticity_2003}, (Fig.\ref{fig:nonaffine-head}) who conclude
that network deformation is non-affine for compliant, low-density networks
and affine for stiff, high-density networks.
 \begin{figure}[ht]
\begin{center}
\includegraphics[width=0.6\textwidth]{Figures/chapter-intro/nonaffine-head.png}%

\caption[Affine and non-affine phases]{ Figure from
\citet{head_mechanical_2005}. Diagram showing the various elastic regimes in
terms of molecular weight (L) and concentration (c). The solid line represents
the rigidity percolation transition, where rigidity first appears at
macroscopic level}
\label{fig:nonaffine-head}
\end{center}
\end{figure}

 Also, \citet{chandran_affine_2006}
showed that non-affinity governs individual fibril kinematics when the load
transfer is from fibril to fibril, with no matrix of fibrils involved.

But non-affinity is challenging to model, and it is the goal of this thesis to provide a method to extract network architecture from real images, to be used as a starting scaffold for the many approaches to simulate dynamics of non-affine regimes, and dynamics with more actors, such as cross-linking molecules.

\section{Network Structure}%
\label{sec:network_structure}

The information needed to fully describe a network structure is \cite{palmer_constitutive_2008}:
\begin{enumerate}[topsep=0pt]
  \item Distribution in the distance between junctions/nodes, i.e. end-to-end distances.
  \item Distribution in the fully extended length of fibers/chains/edges, i.e. contour lengths.
  \item Network connectivity, i.e. degree of nodes, representing the number of neighbors or the number of chains attached to a junction.
  \item Orientation distribution, i.e. angles between adjacent edges.
\end{enumerate}

We will go through the most used methods to emulate network structures used in dynamic simulations that explore strain and stress responses, pointing the simplifications made, and the associated limitations.

\subsection{Mikado Networks}%
\label{sub:mikado_networks}

Ubiquitous in the recent past but still present, is to start the model with a random and homogeneous initialization of the network, such as Mikado models \cite{wilhelm_elasticity_2003, astrom_microscopic_1994, schuster_investigating_2012, conti_cross-linked_2009, sharma_strain-controlled_2016, sharma_elastic_2013},
where fibers of fixed length are thrown into a simulation box until certain density of junctions/intersections is reached (see \autoref{fig:mikado_initial}). Other models used similar random generation, but in 3D \cite{astrom_strain_2008}.
\begin{figure}[ht]
  \centering
  \includegraphics[width=0.4\linewidth]{Figures/chapter-intro/mikado_initial.png}
  \caption{Mikado network used as a starting scaffold for dynamics simulations, from Ref.~\cite{astrom_microscopic_1994}.}
  \label{fig:mikado_initial}
\end{figure}

The distribution of end-to-end distances between junctions can be controlled adding more fibers, but the homogeneous nature of the process inhibits long connections.

The contour length is imposed after the junctions are created, adding arbitrary bending points in the existing chains.

The details of network connectivity are ignored, relying on a mean connectivity between 3 and 4 that is generated in the construction \cite{sharma_strain-controlled_2016}.

The angle distribution is derived from the process, but no extra relation with real networks is provided. Pre-stressing the network can be used to change angles.

\subsection{Lattice Models}%
\label{sub:lattice_models}

Lattice models with different geometries, such as triangular and honey-comb in 2D, and face-centered-cubic (fcc) in 3D have been used \cite{broedersz_filament-length-controlled_2012,ronceray_fiber_2015, vahabi_elasticity_2016, sharma_strain-controlled_2016}. The unrealistic connectivity imposed by the fcc lattice in 3D (see \autoref{fig:fcc_unit_cell}) is trimmed down to experimentally found values by randomly deleting branches.

The end-to-end distances and the contour lengths can be fully controlled, but the connectivity is compromised, the distribution might be too narrow for simple lattices, (honey-comb, triangular), or rely on random dilutions for over-connected lattices (fcc) \cite{sharma_strain-controlled_2016}.

\begin{figure}[ht]
  \centering
  \includegraphics[width=0.3\linewidth]{Figures/chapter-intro/fcc_lattice_wiki.png}
  \caption{Unit cell of a face-centered-cubic lattice, from Ref. \cite{_cubic_2018}}
  \label{fig:fcc_unit_cell}
\end{figure}

\subsection{Eight-chain Models}%
\label{sub:eight_chain_models}

The eight-chain network model \cite{palmer_constitutive_2008, meng_nonlinear_2016} with a extremely simplified network connectivity is able to successfully capture the mechanical properties of non-affine regimes under shear. However, unable to capture other observed behaviours like negative normal stresses \cite{janmey_negative_2006}, or to model networks with cross-links dynamics, and other micro-network events, which are out of its scope.
The eight, or even three-chain models, use non-linear single-chain force-extension curve to model each of its chains, see \autoref{fig:force_extension_CEWLC}. When sheared, the chains can rotate (see \autoref{fig:eight_chain_shear}).

In terms of the network structure, this model is surprisingly effective given its simplicity, the main reason for its success is its ability to accommodate rotation response under shear, instead of only stretching of fibers that would be the only response in the affine regime.

The eight-chain model coupled with single chain non-linear force-extension curves shows a good example of the network architecture have an impact in the mechanical response. Similar simulations, but carried away with a fully detailed network architecture extracted from real samples will be interesting. The network structure of these models are over-simplified:

The distribution of end-to-end distances is a delta, taking a mean distance for each chain, same for the contour length.
The connectivity is eight, or three for the similar three-chain case.
The angle between chains is fixed as well, and can be re-oriented with pre-stress.

% Non-affinity is governed by the micro-architecture of the network, however under shear part of the network architecture can be ignored or is simplified.

\begin{figure}[ht]
  \centering
  \includegraphics[width=0.7\textwidth]{Figures/chapter-intro/eight_chain_shear_nolabels.png}
  \caption{Eight-chain model under shear, from Ref. \citep{palmer_constitutive_2008} }
  \label{fig:eight_chain_shear}
\end{figure}


% TODO more on network micro-architecture.
% As discussed in section \ref{intro-strainstiffening}, at low densities of
% cross-links the network behaves in a non-affine way, where the chain
% straining depends on the local heterogeneity of the media.
% This non-affinity is governed by the micro-architecture of the network
% \citep{head_deformation_2003,wilhelm_elasticity_2003,huisman_three-dimensional_2007,chandran_affine_2006}.

% motivation link
\section{Motivation}%
\label{sec:motivation-intro}
Non-affinity and network architecture are closely related, which
is the \textbf{main motivation} behind this work to stablish analytical methods to study images
from where gather the geometry of networks. This geometry can be used directly in dynamic simulations methods, instead of relying in over-simplified network generators.

The well-established and trusted method to gather structural information of the size of the network
is scattering, particularly small-angle x-ray scattering (SAXS). Information such as porous
size and persistence length are at range of these techniques. However, because its
averaging nature, it is impossible to gather more detailed information about the specific geometry
and connectivity of the network.

So we will rely on gathering structural information directly from images taken from a range microscopy techniques depending on the size of the biopolymer under study.
One of the concerns of this approach, shared specially among the scattering community is how trustable the images are, i.e. if they provide a true representation of the material taken into account that the sample has been modified for the requirements of microscopy preparation. We will answer this, providing a way to compare SAXS data with images, setting the scale at which both techniques give similar result, providing a size range where image features can be trusted.

But before, and in order to study network geometry reliably, we need to provide an image analysis tool-set with algorithms that can be applied to volumetric data, such as tomography and stack of images.
We will extend to 3D, and in a high-performance computing language, state-of-the-art wavelet analysis algorithms, that will aid in our task of gathering the network architecture from images.
We then extract in-silico networks from real 3D images.
These in-silico networks with full details, can be used by theoreticians as starting points for non-affinity modeling, instead of the currently used over-simplified architectures.
We finish providing a method to reconstruct these networks fully in-silico, just with statistical
distributions gathered from the networks extracted from real images. This allow exploration and speculation of optimal network architectures for different processes completely in-silico, but rooted on networks found in nature.

The research question that this work aim to answer is:
\begin{itemize}
  \item How to we use real 3D images to reliably extract realistic network architectures?
\end{itemize}
